\documentclass[12pt]{article}
\usepackage{graphicx} % Required for inserting images


% ===================== PACKAGES =====================
\usepackage[a4paper,margin=1in]{geometry}
\usepackage{amsmath,amssymb}
\usepackage{enumitem}
\usepackage{fancyhdr}
\usepackage{xcolor} 

% ===================== HEADER & FOOTER =====================
\pagestyle{fancy}
\fancyhf{}
\lhead{CSE 400: Fundamentals of Probability in Computing}
\rhead{Lecture - 10 scribe }
\cfoot{\thepage}

% ===================== CUSTOM COMMANDS =====================
\newcommand{\solution}{
    \vspace{0.5cm}
    \noindent\textbf{\textcolor{blue}{Solution:}}
    \vspace{0.2cm}
    
    % --- STUDENT: TYPE YOUR SOLUTION BELOW THIS LINE ---
    
    \vfill 
}


% ===================== TITLE =====================
\title{
    \normalsize School of Engineering and Applied Science (SEAS), Ahmedabad University \\
    \vspace{0.2cm}
    \textbf{CSE 400: Fundamentals of Probability in Computing}\\
    \Large {Lecture - 10 scribe}
}
\author{} 
\date{}

\begin{document}
\maketitle

\vspace{-2cm}
\begin{center}
    \begin{tabular}{ll}
        \textbf{Name:} & {Samridhi Mehrotra \hspace{2.5in}} \\ [1.5ex]
        \textbf{Enrollment No:} & {AU2440033 \hspace{2.5in}} \\ [1.5ex]
        \textbf{Email:} & {samridhi.m@ahduni.edu.in \hspace{2.1in}} \\ [1.5ex]
        
    \end{tabular}
\end{center}

\hrule
\vspace{0.5cm}



\title{\textbf{Lecture 10 : Randomized Min-Cut Algorithm }
\author{}
\date{}

\begin{document}
\maketitle


\section{Overview}

This lecture covers:

\begin{itemize}
    \item Min-Cut Problem
    \item Applications of Min-Cut
    \item Definition of Min-Cut
    \item Successful and Unsuccessful Min-Cut Runs
    \item Max-Flow Min-Cut Theorem
    \item Deterministic Min-Cut Algorithm (Stoer--Wagner)
    \item Randomized Min-Cut Algorithm (Karger)
    \item Pseudocode
    \item Comparison of Deterministic vs Randomized Approaches
    \item Theorem for Min-Cut Probability
    \item Python Simulation Reference
\end{itemize}


\section{Definitions and Notation}

Let $G = (V, E)$ be an undirected graph with:
\begin{itemize}
    \item $V$ = set of vertices
    \item $E$ = set of edges
    \item $|V| = n$
\end{itemize}


\section{Min-Cut Problem}

Given a graph $G = (V, E)$ with $n$ vertices, the \textbf{Minimum Cut (Min-Cut)} problem is to find a cut-set of minimum cardinality.

\subsection{Cut-Set}

A \textbf{cut-set} is a set of edges whose removal disconnects the graph into two or more connected components.

\subsection{Min-Cut}

The \textbf{minimum cut} is the cut-set with the smallest number of edges.


\section{Applications of Min-Cut}

\subsection{Network Design}
Min-cut helps improve communication efficiency and optimize network flow by identifying the minimum capacity cut.

\subsection{Communication Networks}
Min-cut identifies vulnerabilities and helps design fault-tolerant networks.

\subsection{VLSI Design}
In VLSI, min-cut partitions circuits into smaller components to reduce interconnect complexity.

Reference: Section 1.5, \textit{Probability and Computing (2nd Edition)}.


\section{Edge Contraction}

The main operation in Karger's algorithm is \textbf{edge contraction}.

\subsection{Definition}

Contracting an edge $(u, v)$:
\begin{itemize}
    \item Merge vertices $u$ and $v$
    \item Remove the edge $(u,v)$
    \item Remove self-loops
    \item Retain parallel edges
\end{itemize}

The resulting graph may contain parallel edges but no self-loops.


\section{Successful and Unsuccessful Runs}

\subsection{Successful Min-Cut Run}
A successful run correctly identifies the minimum cut after successive contractions.

\subsection{Unsuccessful Min-Cut Run}
An unsuccessful run occurs when the algorithm contracts an edge belonging to the true minimum cut early in the process.

Since the algorithm is randomized, early contraction of critical edges leads to incorrect results.


\section{Max-Flow Min-Cut Theorem}

\subsection{Statement}

In a flow network:

\[
\text{Maximum Flow} = \text{Minimum Cut Capacity}
\]

\subsection{Definitions}

\begin{itemize}
    \item Capacity of a cut: sum of capacities of edges from $X$ to $Y$
    \item Minimum cut: cut with smallest capacity
    \item Maximum flow: largest possible flow from source $S$ to sink $T$
\end{itemize}


\section{Deterministic Min-Cut Algorithm}

\subsection{Stoer--Wagner Algorithm}

Let $s$ and $t$ be vertices of $G$.
Let $G/\{s,t\}$ be the graph obtained by merging $s$ and $t$.

A minimum cut of $G$ is the smaller of:
\begin{itemize}
    \item Minimum $s$-$t$ cut of $G$
    \item Minimum cut of $G/\{s,t\}$
\end{itemize}

\subsection{Proof Idea}

Either:
\begin{itemize}
    \item There exists a minimum cut separating $s$ and $t$
    \item Or no such cut exists, and contraction preserves the minimum cut
\end{itemize}


\section{Pseudocode: Stoer--Wagner}

\subsection*{Algorithm 1: MinimumCutPhase($G$, $a$)}

\begin{verbatim}
A ← {a}
while A ≠ V do
    add to A the most tightly connected vertex
return cut weight as the cut-of-the-phase
\end{verbatim}

\subsection*{Algorithm 2: MinimumCut($G$)}

\begin{verbatim}
while |V| ≥ 1 do
    choose any vertex a
    MinimumCutPhase(G, a)
    if cut-of-the-phase < current minimum
        update minimum cut
    merge the last two vertices added
return minimum cut
\end{verbatim}

\subsection{Time Complexity}

\[
O(V \cdot E + V^2 \log V)
\]


\section{Randomized Min-Cut Algorithm}

\subsection{Why Randomized?}

\begin{itemize}
    \item Probabilistic success guarantee
    \item Efficient for large graphs
    \item Avoids worst-case deterministic behavior
    \item Easily parallelizable
\end{itemize}


\section{Karger’s Randomized Algorithm}

\subsection{Algorithm Idea}

\begin{itemize}
    \item Randomly select an edge
    \item Contract it
    \item Repeat until only 2 vertices remain
    \item Remaining edges form the cut
\end{itemize}

\subsection{Example}

Example run:
\begin{itemize}
    \item Pick edge $b$, contract
    \item Pick edge $d$, contract
    \item Output cut: $\{a,c,e\}$
\end{itemize}

True minimum cut:
\[
\{b,e\} \quad \text{or} \quad \{a,d\}
\]


\section{Recursive Randomized Min-Cut}

\subsection*{Algorithm 3: Recursive-Randomized-Min-Cut($G$, $\alpha$)}

\begin{verbatim}
if n ≤ α^3 then
    return brute-force min-cut
else
    for i = 1 to α do
        G' ← apply random contractions
        C' ← Recursive-Randomized-Min-Cut(G', α)
        keep smallest cut found
return C
\end{verbatim}

\subsection{Time Complexity}

\[
O(V^2)
\]


\section{Probability Theorem for Min-Cut}

The algorithm outputs a minimum cut with probability at least:

\[
\frac{2}{n(n-1)}
\]

This probability increases with repeated independent runs.


\section{Comparison: Deterministic vs Randomized}

\subsection{Deterministic (Stoer--Wagner)}

\begin{itemize}
    \item Always exact
    \item Higher time complexity
\end{itemize}

\subsection{Randomized (Karger)}

\begin{itemize}
    \item Approximate with high probability
    \item Lower time complexity
    \item Requires repetition for higher confidence
\end{itemize} 


\section{Derivations / Proof Summary}

Stoer--Wagner correctness follows from:

\begin{itemize}
    \item If minimum cut separates $s$ and $t$, then minimum $s$-$t$ cut is global minimum.
    \item Otherwise, contraction preserves minimum cut.
\end{itemize}


\section{Python Simulation}

Steps:
\begin{itemize}
    \item Open Campuswire Post for Lecture 10
    \item Download provided \texttt{.ipynb} file
    \item Run simulation of Karger’s algorithm
\end{itemize}


\section{Summary}

\begin{itemize}
    \item Cut-set: edges whose removal disconnects graph.
    \item Min-cut: minimum cardinality cut-set.
    \item Edge contraction: merge vertices, remove self-loops.
    \item Max-flow equals min-cut capacity.
    \item Stoer--Wagner: deterministic, $O(VE + V^2 \log V)$.
    \item Karger: randomized, $O(V^2)$.
    \item Probability of success $\ge \frac{2}{n(n-1)}$.
\end{itemize}

\end{document}

