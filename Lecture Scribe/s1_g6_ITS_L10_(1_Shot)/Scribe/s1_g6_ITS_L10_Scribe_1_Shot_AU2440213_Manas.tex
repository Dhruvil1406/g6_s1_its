\documentclass[11pt]{article}
\usepackage[utf8]{inputenc}
\usepackage[margin=1in]{geometry}
\usepackage{amsfonts, amsmath, amssymb}
\usepackage{graphicx}
\usepackage{hyperref}
\usepackage{enumitem}

\begin{document}

% --- Header Section ---
\begin{center}
    \textbf{\Large School of Engineering and Applied Science (SEAS), Ahmedabad University} \\
    \textbf{CSE 400: Fundamentals of Probability in Computing} \\
    \textbf{Lecture 10: Randomized Min-Cut Algorithm} \\
    \rule{\linewidth}{0.5pt}
\end{center}

\noindent \textbf{Lecturer:} Dhaval Patel, PhD \hfill \textbf{Date:} February 5, 2026 \\
\textbf{Topic:} Randomized Min-Cut Algorithm \hfill \textbf{Scribe:} Lecture Summary

\vspace{1em}

% --- Section 1: Outline ---
\section{Outline}
This lecture covers the fundamental concepts of the Min-Cut problem, comparing deterministic and randomized approaches. Key topics include:
\begin{itemize}[noitemsep]
    \item The Min-Cut Problem: Definition and Applications.
    \item Max-Flow Min-Cut Theorem.
    \item Deterministic Min-Cut (Stoer-Wagner Algorithm).
    \item Randomized Min-Cut (Karger's Algorithm).
    \item Performance Comparison and Success Probabilities.
\end{itemize}

\section{The Min-Cut Problem}

\subsection{Why use Min-Cut?}
Min-cut algorithms are utilized in various domains to solve problems related to network connectivity, reliability, and optimization:
\begin{itemize}
    \item \textbf{Network Design:} Improves communication efficiency and optimizes network flow by finding the minimum capacity cut.
    \item \textbf{Communication Networks:} Helps understand network vulnerability to failures and aids in building robust, fault-tolerant systems.
    \item \textbf{VLSI Design:} Useful for partitioning circuits into smaller components to reduce interconnectivity complexity.
\end{itemize}

\subsection{What is a Min-Cut?}
\begin{itemize}
    \item \textbf{Cut-set:} A set of edges whose removal breaks a graph into two or more connected components.
    \item \textbf{Min-cut Problem:} Given a graph $G = (V, E)$ with $n$ vertices, find a minimum cardinality cut-set in $G$.
\end{itemize}

\subsection{Edge Contraction}
The primary operation in randomized min-cut algorithms is \textbf{edge contraction}:
\begin{itemize}
    \item It removes an edge $(u, v)$ and merges vertices $u$ and $v$ into a single vertex.
    \item All edges connecting $u$ and $v$ are eliminated.
    \item All other edges are retained; the resulting graph may have parallel edges but no self-loops.
\end{itemize}

\section{Success and Failure in Min-Cut Runs}
\begin{itemize}
    \item \textbf{Successful Run:} An iteration of the algorithm that correctly identifies the minimum cut of the graph.
    \item \textbf{Unsuccessful Run:} An iteration where the algorithm fails to identify the minimum cut. This can happen if the algorithm happens to contract critical edges (edges belonging to the min-cut) early in the process.
\end{itemize}

\section{Max-Flow Min-Cut Theorem}
The theorem states: ``In a flow network, the maximum amount of flow passing from the source ($S$) to the sink ($T$) is equal to the total weight of the edges in a minimum cut.''
\begin{itemize}
    \item \textbf{Capacity of a Cut:} The sum of the capacity of edges oriented from a vertex in set $X$ to a vertex in set $Y$.
    \item \textbf{Minimum Cut:} The cut in the network with the smallest possible capacity.
    \item \textbf{Maximum Flow:} The largest possible flow from source $S$ to sink $T$.
\end{itemize}

\section{Deterministic Min-Cut Algorithm}
\subsection{Stoer-Wagner Algorithm}
This algorithm provides an exact solution. The core theorem states that for two vertices $s$ and $t$, a minimum cut of $G$ is the smaller of:
\begin{enumerate}
    \item A minimum $s-t$-cut of $G$.
    \item A minimum cut of $G/\{s, t\}$ (the graph where $s$ and $t$ are merged).
\end{enumerate}

\paragraph{Complexity:} $O(V \cdot E + V^2 \log V)$.

\section{Randomized Min-Cut Algorithm}
\subsection{Why Randomized Algorithms?}
\begin{itemize}
    \item Provide a probabilistic guarantee of success.
    \item Often provide accurate estimates with fewer iterations.
    \item Advantages: Efficiency, Parallelization, Approximation Guarantees, Robustness, and avoidance of worst-case instances.
\end{itemize}

\subsection{Karger's Randomized Algorithm}
Karger’s algorithm uses random contraction steps to find a cut.
\paragraph{Theorem for Min-Cut Set:} The algorithm outputs a min-cut set with probability at least:
$$P(\text{success}) \ge \frac{2}{n(n-1)}$$

\paragraph{Complexity:} $O(V^2)$.

\section{Comparison: Deterministic vs. Randomized}
\begin{table}[h!]
\centering
\begin{tabular}{|l|l|l|}
\hline
\textbf{Feature} & \textbf{Deterministic (Stoer-Wagner)} & \textbf{Randomized (Karger's)} \\ \hline
\textbf{Guarantee} & Always finds exact minimum cut & Approximate min-cut with high probability \\ \hline
\textbf{Complexity} & $O(V \cdot E + V^2 \log V)$ & $O(V^2)$ \\ \hline
\textbf{Scalability} & Higher complexity for large graphs & Generally more efficient for large graphs \\ \hline
\end{tabular}
\end{table}

\section{Python Simulation}
A simulation of the Randomized Min-Cut algorithm can be found in the lecture 10 Campuswire post (File: \texttt{L8\_RandomizedMin...ipynb}).

\vspace{2em}
\begin{center}
    \textbf{End of Scribe}
\end{center}

\end{document}