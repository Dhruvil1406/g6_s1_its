\documentclass[11pt]{article}

\usepackage{amsmath, amssymb, amsthm}
\usepackage{graphicx}
\usepackage{enumitem}
\usepackage{hyperref}
\usepackage{geometry}
\geometry{margin=1in}

\title{CSE 400: Fundamentals of Probability in Computing\\
Lecture 10: Randomized Min-Cut Algorithm\\
Group-6 Scribe Refactoring}

\date{February 5, 2026}

\begin{document}

\maketitle

\section{Introduction}

This lecture studies the \textbf{Minimum Cut (Min-Cut)} problem in graphs,
including deterministic and randomized approaches.

Applications include:
\begin{itemize}
    \item Network design and optimization
    \item Communication network reliability
    \item VLSI circuit partitioning
\end{itemize}

\section{Min-Cut Problem}

\subsection{Definition of a Cut}

Given a graph $G = (V,E)$:

A \textbf{cut-set} is a set of edges whose removal disconnects the graph into two or more components.

\subsection{Minimum Cut}

The \textbf{minimum cut} of $G$ is a cut-set of minimum cardinality.

Formally:

\[
\text{Min-Cut}(G) = \min_{S \subset V} |E(S, V \setminus S)|
\]

where $E(S, V \setminus S)$ is the set of edges crossing the partition.

\section{Edge Contraction}

The core operation in many min-cut algorithms is \textbf{edge contraction}.

When contracting edge $(u,v)$:

\begin{itemize}
    \item Merge vertices $u$ and $v$
    \item Remove self-loops
    \item Retain parallel edges
\end{itemize}

After repeated contractions, the graph reduces in size.

\section{Max-Flow Min-Cut Theorem}

\textbf{Theorem:}  
In a flow network, the maximum flow from source $S$ to sink $T$
is equal to the capacity of the minimum cut.

\subsection{Definitions}

\begin{itemize}
    \item Capacity of cut: Sum of capacities of edges from $X$ to $Y$
    \item Minimum cut capacity: Smallest such capacity
    \item Maximum flow: Largest feasible flow from $S$ to $T$
\end{itemize}

\section{Deterministic Min-Cut Algorithm}

\subsection{Stoer-Wagner Algorithm}

Let $s,t$ be vertices in $G$.

A minimum cut of $G$ is the smaller of:

\begin{itemize}
    \item Minimum $s$-$t$ cut of $G$
    \item Minimum cut of $G/\{s,t\}$
\end{itemize}

\subsection{Pseudocode}

\textbf{Algorithm 1: MinimumCutPhase(G, a)}

\begin{verbatim}
A ← {a}
while A ≠ V do
    add most tightly connected vertex to A
return cut weight
\end{verbatim}

\textbf{Algorithm 2: MinimumCut(G)}

\begin{verbatim}
while |V| ≥ 1 do
    choose a ∈ V
    MinimumCutPhase(G, a)
    update minimum cut if needed
    merge last two added vertices
return minimum cut
\end{verbatim}

\subsection{Time Complexity}

Stoer-Wagner runs in:

\[
O(VE + V^2 \log V)
\]

\section{Randomized Min-Cut Algorithm}

\subsection{Why Randomization?}

\begin{itemize}
    \item Avoid worst-case inputs
    \item Faster in practice
    \item Easier parallelization
    \item Approximation guarantees
\end{itemize}

\section{Karger's Randomized Algorithm}

\subsection{Algorithm Idea}

\begin{enumerate}
    \item While $|V| > 2$:
    \begin{itemize}
        \item Choose random edge
        \item Contract it
    \end{itemize}
    \item Return remaining cut
\end{enumerate}

\subsection{Time Complexity}

\[
O(V^2)
\]

\section{Probability of Success}

\textbf{Theorem:}

Karger's algorithm outputs a minimum cut with probability at least

\[
\frac{2}{n(n-1)}
\]

\subsection{Proof Sketch}

Let the minimum cut size be $k$.

At each contraction step:

\[
P(\text{avoid min-cut edge}) \ge 1 - \frac{k}{\text{total edges}}
\]

After $n-2$ contractions:

\[
P(\text{success}) \ge \prod_{i=0}^{n-3}
\left(1 - \frac{2}{n-i}\right)
= \frac{2}{n(n-1)}
\]

Thus repeating algorithm multiple times increases success probability.

\section{Comparison}

\begin{center}
\begin{tabular}{|c|c|c|}
\hline
Property & Deterministic & Randomized \\
\hline
Guarantee & Exact & High Probability \\
Time Complexity & $O(VE + V^2 \log V)$ & $O(V^2)$ \\
Worst Case & Predictable & Probabilistic \\
Parallelization & Limited & Easy \\
\hline
\end{tabular}
\end{center}

\section{Python Simulation}

A simulation notebook was provided for experimentation with Karger’s algorithm.

Students are encouraged to:
\begin{itemize}
    \item Run multiple trials
    \item Observe probability convergence
    \item Compare deterministic vs randomized performance
\end{itemize}

\section{Exam-Oriented Summary}

\begin{itemize}
    \item Understand definition of cut and minimum cut.
    \item Know edge contraction clearly.
    \item Be able to state Max-Flow Min-Cut theorem.
    \item Write Stoer-Wagner pseudocode.
    \item Write Karger’s algorithm.
    \item Derive probability $\frac{2}{n(n-1)}$.
    \item Compare deterministic vs randomized approaches.
    \item Know time complexities.
\end{itemize}

\section{Key Takeaways}

\begin{itemize}
    \item Min-cut measures graph connectivity.
    \item Deterministic algorithms guarantee exact solutions.
    \item Randomized algorithms trade certainty for efficiency.
    \item Repetition improves randomized success probability.
\end{itemize}

\end{document}
